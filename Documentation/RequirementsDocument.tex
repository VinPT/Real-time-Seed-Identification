\documentclass[onecolumn, draftclsnofoot,10pt, compsoc]{IEEEtran}
\usepackage{graphicx}
\usepackage{url}
\usepackage{setspace}

\usepackage{geometry}
\geometry{textheight=9.5in, textwidth=7in}

% 1. Fill in these details
\def \CapstoneTeamName{			Green Team}
\def \CapstoneTeamNumber{		11}
\def \GroupMemberOne{			Omar Elgebaly}
\def \GroupMemberTwo{			Danny Yang}
\def \GroupMemberThree{			Vinayaka Thompson}
\def \CapstoneProjectName{		Real-time Seed Identification}
\def \CapstoneSponsorCompany{	Oregon State University Crop Science Department}
\def \CapstoneSponsorPerson{	Daniel Curry}

% 2. Uncomment the appropriate line below so that the document type works
\def \DocType{		Requirements Document
				%Requirements Document
				%Technology Review
				%Design Document
				%Progress Report
				}

\newcommand{\NameSigPair}[1]{\par
\makebox[2.75in][r]{#1} \hfil 	\makebox[3.25in]{\makebox[2.25in]{\hrulefill} \hfill		\makebox[.75in]{\hrulefill}}
\par\vspace{-12pt} \textit{\tiny\noindent
\makebox[2.75in]{} \hfil		\makebox[3.25in]{\makebox[2.25in][r]{Signature} \hfill	\makebox[.75in][r]{Date}}}}
% 3. If the document is not to be signed, uncomment the RENEWcommand below
\renewcommand{\NameSigPair}[1]{#1}

%%%%%%%%%%%%%%%%%%%%%%%%%%%%%%%%%%%%%%%
\begin{document}
\begin{titlepage}
    \pagenumbering{gobble}
    \begin{singlespace}
    	%\includegraphics[height=4cm]{coe_v_spot1}
        \hfill
        % 4. If you have a logo, use this includegraphics command to put it on the coversheet.
        %\includegraphics[height=4cm]{CompanyLogo}
        \par\vspace{.2in}
        \centering
        \scshape{
            \huge CS Capstone \DocType \par
            {\large\today}\par
			{\large CS461 Fall 2017}\par
            \vspace{.5in}
            \textbf{\Huge\CapstoneProjectName}\par
            \vfill
            {\large Prepared for}\par
            \Huge \CapstoneSponsorCompany\par
            \vspace{5pt}
            {\Large\NameSigPair{\CapstoneSponsorPerson}\par}
            {\large Prepared by }\par
            Group\CapstoneTeamNumber\par
            % 5. comment out the line below this one if you do not wish to name your team
            \CapstoneTeamName\par
            \vspace{5pt}
            {\Large
                \NameSigPair{\GroupMemberOne}\par
                \NameSigPair{\GroupMemberTwo}\par
                \NameSigPair{\GroupMemberThree}\par
            }
            \vspace{20pt}
        }
        \begin{abstract}
        % 6. Fill in your abstract
        	The purpose of this project is to develop a solution that will accurately differentiate between desired grass seed and unwanted grass seed. The OSU Crop Science department are given batches of 25,000 seeds at a time. These batches are received from various clients such as golf courses, gardens, etc. When the Crop Science departments receive these seeds, there are off-types embedded within those packets that need to be removed. The reasoning behind this is because the off-types can contaminate the batch and cause weeds and other undesirable plants to sprout. The Crop Science department’s job is to pull out all the off-type seeds by hand through visual inspection. The seeds are magnified and the analysts differentiate between the good seed and the bad seed by looking at different markers such as shape, texture, color, etc. This is an extremely time-consuming process that is tedious and causes eyestrain, headaches, and backaches.
Our job is to utilize a machine learning tool that can identify the off-type seeds as off-types so they can be removed from the batch. The challenge is that the classifier must be able to remove every off-type. According to our client, the batches are generally 99 percent desired type and 1 percent off-type. This means we need to implement an algorithm with a relatively low margin of error that will accurately identify all the off-types.


        \end{abstract}
    \end{singlespace}
\end{titlepage}
\newpage
\pagenumbering{arabic}
\tableofcontents
% 7. uncomment this (if applicable). Consider adding a page break.
%\listoffigures
%\listoftables
\clearpage

% 8. now you write!
\section{Definition and Description of Problem}
The OSU Crop Science Department takes batches of 25,000 grass seed at a time from clients. The department is paid to sort through the seed and separate the desired seed from the off-types, which will unavoidably be mixed in. The reasoning behind taking out these off-types is because a client such as a golf course owner will desire a specific type of grass seed to be planted. If the off-types are planted with the grass seed, it could potentially result in contamination and the spreading of weeds.
The Crop Science Department has analysts that perform this sorting job by hand. They sit for hours a day and differentiate between the off-types and the desired seed by looking at texture, shape, and color with the aid of a high-powered lens. This is a tedious process that causes eyestrain, headaches, and backaches. It is also inefficient because it is not only slow but also prone to human error, especially after sifting through seeds for long periods of time.

\section{Proposed Solution}
Our proposed solution involves using a high-powered lens camera to capture and train images in a classifier. First, we will capture 25,000 images of the desired seed, Tall Fescue. The next step is to normalize the data in such a way that it reduces redundancy and getting all the data on the same scale. The goal here is to train the images with a margin of error of ~25 percent. This means when the algorithm takes in an image of Tall Fescue seed as input, it should do nothing. When the algorithm takes in an image of an off-type grass seed as input, it should flag it as an off-type.

\section{}
A complete version of this project would ideally be able to identify seeds of the desired grass type in real-time or more accurately identify what is not the desired grass seed type with a 25% margin of error. It is imperative our margin of error is not too high because too many off-types slipping through the cracks could cause contamination in the Crop Science department’s client, which could cause them to lose business. Our client mentioned they would rather throw away good seed than let off-types contaminate the batch. This means our leeway for false positives won’t be as rigorous as previously thought
In addition to having a high accuracy, the software needs to be able to identify the seeds in real time. This means the testing phase of the machine learning algorithm should be able to accurately identify and flag off-types. Everything beyond that is up to the MIME group that is creating the conveyor belt and hooking it up to the computer.
There are many potential challenges with this project. The first challenge I see is that we may not have the resources to feasibly get a perfect accuracy. Between our group, we have one Nvidia GPU. Even with that, training a batch of 25,000 images could take up to 6 days. We need to figure out exactly what kind of requirements can be expected with the hardware we have.
Another notable challenge I see is the speed at which the classifier will be able to test the real-time images of the seeds being inputted. Our client expects to have a conveyor belt, with a camera suspended in the middle. It is hooked up to a computer and constantly uploading images of seeds to be tested. If the classifier classified the seed as an off-type, it is flagged and the computer will send a signal to the conveyor belt, which will indicate it should stop. Although the conveyor belt is not our responsibility, the testing phase needs to be quick enough that it can keep up with the belt.
Finally, one of the biggest challenges is in the data collection phase. The grass seed is incredibly small, which means that a camera with a high-powered lens is required. Any visual noise could drastically affect accuracy. Furthermore, the normalization of the data could prove problematic if our images are not sufficiently clear. It is imperative we choose the right machine learning tool for this project as we do not have the time to spend a week training. Tensorflow is an attractive candidate for this job but it is implemented in Python, which leads me to believe that training could take too long to be feasible. Finding the right tool is imperative as we need to something that can process and train our images in a feasible window of time if we are to get accurate results.



1 INTRODUCTION

2 OVERALL DESCRIPTION

3 SPECIFIC REQUIREMENTSS


\section{Introduction}
The purpose of this project is to identify off-type seeds in real-time.
The seeds go through a conveyor belt with a camera constantly feeding inuges to the classifier via a USB connection to a computer.
If an off-type is detected, the computer will send a signal and the conveyor belt will trash any seeds in the picture involving a bad seed, The client allows false negatives but limits false positives.
The Scope Of this project involves implementing a machine learning to identify seeds in real-time.
Anything beyond the software is out Of this project's scope.
The classifier Should binary.
This means It only has to determine if a seed IS an off-type or desired type, Which is Tall Fescue In this case.
 \section{Purpose}
In this world there is a need to sort seed.
This could be for many reasons from a farmer wanting to take bad seed out of the seed he is saving, to a lab tech tasked with categorizing seed.
In the case of this project we are working with the osu seed lab.
They are sent many packets of random seed and are tasked with grading it.
This process of grading seed requires researchers to go through the seed on by one under a microscope to determine the seed type and sort it.
When doing this usually 90 percent or more of the seed is good seed.
Because the lab tech needs to sort these seeds by hand it is a very taxing job.
Thus, our goal is to remove 80% of the good seed so that the seed given to the lab tech is the seed he is interested in.
Making it so that he is only focusing on the seed that is important to his job.

The machine learning algorithm should be able to accurately identify off-types with a 20% margin of error.
The testing phase of the product should process 14 seeds per second.
This means it should be able to process a batch of 25,000 seeds in approximately 30 minutes.
The user expressed interest in an interface that will allow them to train images of other desired seed.
There are two major constraints for this project: time and accuracy Accuracy is easily the most
important here as allowing too many off-types into the good seed batch Will make the batch unusable.
At the minimum, the classifier should be able to identify good seed with a 100% aculracy.
The other constraint is time.
A batch of seeds should take about 30 minutes to complete.
The client mentioned they are more flexible In this regard.
He also mentioned that accuracy should not be sacrificed to increase speed.
 \section{Scope}
The scope of this project is removing good grass seed from bad grass and weed seed.
The projects scope only requires we be able to sort one type of grass seed out of the others and that the seeds we are sorting will look no similar than perennial rye grass and tall fescue.

The machine learning algorithm should be able to accurately identify off-types with a margin of error.
The testing phase Of the product should process 14 seeds per Second.
This it be able to process a batch Of 25,000 seeds in approximately 30 minutes. are the hard requirements for the project.
The Client expressed interest in an interface, which would allow him to train images of other desired seeds.
It was decided that this would be implemented as a STRETCH goal.
The client also expressed interest in categorizing the off-type seeds.
It was also agreed that this would be a STRETCH goal and would be implemented if time constraints are not too tight. 
 \section{Definitions acronyms and abbreviations}
Client: In this case our client is OSU Seed Lab
Labtech: someone who sorts and identifies seed for OSU Seed Lab
Researcher: someone who works at OSU Seed Lab
OSU: Oregon State University
Over all description
\section{Product functions}
The primary function of this product to sort grass seed and remove all the bad or undesirable seed (in this case tall fescue and perennial rye grass are the hardest two we will need to sort and make it the focus of this project).
In the future the client expressed a desire to have the device determine the type of each seed running through it however as they require 100% success this is something beyond the scope of this project.
\section{User characteristics}
The use case for this device is twofold.
The first and primary user is the Seed lab.
Where the user would be some sort of lab tech who would be using the device to reduce their work.
The other possible user is a seed enthusiast in the public.
The client said somethings about wanting to in the future expand the project into some sort of product that could be used by anyone who wanted to sort seed.
\section{Constraints}
This project has a few true constraints provided by the client.
These constraints are we must have no false positives. We must be able to remove at least 80% of the good seed from the bad seed.
We must identify about 14 seeds per second.
In addition to this the client expressed a desire for some additional systems in the future and wanted us to take that in consideration when building our system.
These future projects are, allowing the client to change the seed sorted on the device by allowing the client to train it or get an update to the device that will train the device and the other possible expiation project was to identify the bad seed that was removed from the good seed.
\section{Assumptions and dependencies}
Though our project has some equipment we can use from a previous project the client seems to be wanting us to start from scratch.
That said we do have some access to a camera system the last group was able to setup and get working.
The assumption is being made that we don’t need to build any of the hardware components.
We are also assuming that we only need to send a true false signal to the hardware team.
Finally, we assume that the hardware specs are what we specify regardless of what the mechanical team makes.


\end{document}
