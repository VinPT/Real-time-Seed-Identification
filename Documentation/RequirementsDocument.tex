\documentclass[onecolumn, draftclsnofoot,10pt, compsoc]{IEEEtran}
\usepackage{graphicx}
\usepackage{url}
\usepackage{setspace}
\usepackage{pgfgantt}
\usepackage{titlesec}

\usepackage{geometry}
\geometry{textheight=9.5in, textwidth=7in}

% 1. Fill in these details
\def \CapstoneTeamName{			Green Team}
\def \CapstoneTeamNumber{		11}
\def \GroupMemberOne{			Omar Elgebaly}
\def \GroupMemberTwo{			Danny Yang}
\def \GroupMemberThree{			Vinayaka Thompson}
\def \CapstoneProjectName{		Real-time Seed Identification}
\def \CapstoneSponsorCompany{	Oregon State University Crop Science Department}
\def \CapstoneSponsorPerson{	Daniel Curry}

% 2. Uncomment the appropriate line below so that the document type works
\def \DocType{		Requirements Document
				%Requirements Document
				%Technology Review
				%Design Document
				%Progress Report
				}

\newcommand{\NameSigPair}[1]{\par
\makebox[2.75in][r]{#1} \hfil 	\makebox[3.25in]{\makebox[2.25in]{\hrulefill} \hfill		\makebox[.75in]{\hrulefill}}
\par\vspace{-12pt} \textit{\tiny\noindent
\makebox[2.75in]{} \hfil		\makebox[3.25in]{\makebox[2.25in][r]{Signature} \hfill	\makebox[.75in][r]{Date}}}}
% 3. If the document is not to be signed, uncomment the RENEWcommand below
\renewcommand{\NameSigPair}[1]{#1}

%%%%%%%%%%%%%%%%%%%%%%%%%%%%%%%%%%%%%%%
\begin{document}
\begin{titlepage}
    \pagenumbering{gobble}
    \begin{singlespace}
    	%\includegraphics[height=4cm]{coe_v_spot1}
        \hfill
        % 4. If you have a logo, use this includegraphics command to put it on the coversheet.
        %\includegraphics[height=4cm]{CompanyLogo}
        \par\vspace{.2in}
        \centering
        \scshape{
            \huge CS Capstone \DocType \par
            {\large\today}\par
			{\large CS461 Fall 2017}\par
            \vspace{.5in}
            \textbf{\Huge\CapstoneProjectName}\par
            \vfill
            {\large Prepared for}\par
            \Huge \CapstoneSponsorCompany\par
            \vspace{5pt}
            {\Large\NameSigPair{\CapstoneSponsorPerson}\par}
            {\large Prepared by }\par
            Group\CapstoneTeamNumber\par
            % 5. comment out the line below this one if you do not wish to name your team
            \CapstoneTeamName\par
            \vspace{5pt}
            {\Large
                \NameSigPair{\GroupMemberOne}\par
                \NameSigPair{\GroupMemberTwo}\par
                \NameSigPair{\GroupMemberThree}\par
            }
            \vspace{20pt}
        }
        \begin{abstract}
        % 6. Fill in your abstract
				The purpose of this project is to identify off-type seeds in real-time.
				The seeds go through a conveyor belt with a camera constantly feeding inuges to the classifier via a USB connection to a computer.
				If an off-type is detected, the computer will send a signal and the conveyor belt will trash any seeds in the picture involving a bad seed, The client allows false negatives but limits false positives.
				The Scope Of this project involves implementing a machine learning to identify seeds in real-time.
				Anything beyond the software is out Of this project's scope.
				The classifier Should binary.
				This means It only has to determine if a seed IS an off-type or desired type, Which is Tall Fescue In this case.

        \end{abstract}
    \end{singlespace}
\end{titlepage}
\newpage
\pagenumbering{arabic}
\tableofcontents
% 7. uncomment this (if applicable). Consider adding a page break.
%\listoffigures
%\listoftables
\clearpage

% 8. now you write!

\section{Introduction}

\subsection{Purpose}
In the world today there is a need to sort seed.
Resons for sorting seed may include grading for pricing pourposes or cleaning for reseeding.
This is an arduous task requireing, for example a researcher to go through the seed on by one under a microscope to determine the seed type and sort it.
The difficulty is compounded when working will small seeds like those produced by grass plants and usually 90 percent or more of the seed is good seed.
Thus we are working on a device to reduce the work required to sort seed with a focus on sorting grass seed.
The product we are working on is being designed for the OSU Seed Lab, making the labtechs working for them our primary audence.
There is however a secondary group that this device could be targeted to namely farmers needing to clean small batches of seed for reseeding and other such usecases.

\subsection{Scope}
The product to be created by our team is a program that is to work with a theoretical set of hardware components to sort seed.
This program is designed to take a photo of the seed (in this case grass) process it localy and then send to the device signal to the hardware.
The algorith we make should be able to accurately identify alen seed within an minimum 80% margin of error.
When testing seed the fully assembled device should be able to process about 14 seeds per Second.
The client would like it to be able to process a batch Of 25,000 seeds in approximately 30 minutes.
Additionaly there was a desire that the device have some sort of user interface which would allow the user to change values and get feedback.
When working on the above program we need to build it in such a way that the folowing extentions to our product are possible.
The first is the ability for the user to train the device or be able to switch seed type through some user installable upgrade.
The second was caegorising the oftype seed thus giving an estimation of what types of alen seeds are present in the sample.
Should we accomplish all of the above requirements to the best of our ability with significant time left before the deadline the client would like us to start on one of the two extention projects.

\subsection{Definitions acronyms and abbreviations}
Client: In this case our client is OSU Seed Lab

Labtech: Someone who sorts and identifies seed for OSU Seed Lab.

Researcher: Someone who works at OSU Seed Lab.

OSU: Oregon State University

\subsection{Overview}
unfinished

\section{Overall description}

\subsection{Product perspective}

\subsubsection{System Interfaces}
The software is to be built on a
unfinished
\subsubsection{User Interfaces}
The user will be interacting witht the product in 2 primary ways.
The first is quite simple and simply re
un finished
\subsubsection{Software Interfaces}
Our Product will have four software interfaces.
The two interfaces are to designed to connect to the two expantion projects and the other two are dependencys of the project.
This project is being built on a linux device so that we will not have to deal with low level programming.
Additionaly the product will be using a libarary to help us with machene vision
To allow the interface expantion and changing the seed we are sorting we are simpley exposing data that wiould be usefull in that case when building our classes.
To clasify the bad seed we are exposing all of the bad seed pictures and leaving them in a file on the device such that it can be accessed easyly
\subsubsection{Product perspective}
unfinished

\subsubsection{Memory constraints}
unfinished
\subsubsection{Operations}
The basic opperations of this device are as simple as turing it on.
This will not be the case if someone creates the expanshon project.



\subsection{Product functions}
The primary function of this product to sort grass seed and remove all the bad or undesirable seed (in this case tall fescue and perennial rye grass are the hardest two we will need to sort and make it the focus of this project).
In the future the client expressed a desire to have the device determine the type of each seed running through it however as they require 100% success this is something beyond the scope of this project.

\subsection{User characteristics}
The use case for this device is twofold.
The first and primary user is the Seed lab.
Where the user would be some sort of lab tech who would be using the device to reduce their work.
The other possible user is a seed enthusiast in the public.
The client said somethings about wanting to in the future expand the project into some sort of product that could be used by anyone who wanted to sort seed.

\subsection{Constraints}
This project has a few true constraints provided by the client.
These constraints are we must have no false positives. We must be able to remove at least 80% of the good seed from the bad seed.
We must identify about 14 seeds per second.
In addition to this the client expressed a desire for some additional systems in the future and wanted us to take that in consideration when building our system.
These future projects are, allowing the client to change the seed sorted on the device by allowing the client to train it or get an update to the device that will train the device and the other possible expiation project was to identify the bad seed that was removed from the good seed.

\subsection{Assumptions and dependencies}
Though our project has some equipment we can use from a previous project the client seems to be wanting us to start from scratch.
That said we do have some access to a camera system the last group was able to setup and get working.
The assumption is being made that we don’t need to build any of the hardware components.
We are also assuming that we only need to send a true false signal to the hardware team.
Finally, we assume that the hardware specs are what we specify regardless of what the mechanical team makes.
\subsection{Apportioning of requirements}




The machine learning algorithm should be able to accurately identify off-types with a 20% margin of error.
The testing phase of the product should process 14 seeds per second.
This means it should be able to process a batch of 25,000 seeds in approximately 30 minutes.
The user expressed interest in an interface that will allow them to train images of other desired seed.
There are two major constraints for this project: time and accuracy Accuracy is easily the most
important here as allowing too many off-types into the good seed batch Will make the batch unusable.
At the minimum, the classifier should be able to identify good seed with a 100% aculracy.
The other constraint is time.
A batch of seeds should take about 30 minutes to complete.
The client mentioned they are more flexible In this regard.
He also mentioned that accuracy should not be sacrificed to increase speed.

\begin{ganttchart}{1}{30}
\gantttitlelist{2017,2018}{10} \\
\gantttitlelist{"Fall","Winter","Spring"}{10} \\
\gantttitlelist{1,...,30}{1} \\
%\ganttgroup{Group 1}{1}{7} \\
\ganttbar{Reseach}{1}{6} \\
\ganttbar{Camera Testing}{5}{8}
\ganttbar{Algorithm Testing}{5}{10}
\ganttbar{Algorithm Implementation}{10}{23} \ganttnewline
\ganttbar{Expantion Project}{20}{23} \ganttnewline
\ganttmilestone{Be Done}{23} \ganttnewline


%\ganttlink{elem1}{elem3}



\end{ganttchart}


\end{document}
